% !TEX program = xelatex
% !TEX encoding = UTF-8
% !BIB program = bibtex
\documentclass[12pt,a4paper]{article}

% --- 版面與中文字 ---
\usepackage{geometry}
\geometry{a4paper,margin=1in}

\usepackage{fontspec}
\usepackage[BoldFont]{xeCJK}
% 統一中英文字體(可按需更換)
\setCJKmainfont{標楷體}
\setmainfont{Times New Roman}

% 中文自動換行+調整 CJK glue,減少不均勻間隙
\XeTeXlinebreaklocale "zh"
\XeTeXlinebreakskip = 0pt plus 1pt
\xeCJKsetup{CJKglue=\hspace{0pt plus 0.08em minus 0.02em}}

% 段落樣式:縮排+微小段前後距,避免空白忽大忽小
\setlength{\parindent}{2em}
\setlength{\parskip}{0.35em}
\linespread{1.2}

% 清單間距統一
\usepackage{enumitem}
\setlist{nosep,leftmargin=2em}

% 數學
\usepackage{amsmath,amssymb}

% --- 參考文獻格式 ---
\usepackage{cite}

% --- 封面資訊 ---
\newcommand{\HomeworkTitle}{Image Processing Homework \#1: Spatial Image Enhancement}
\newcommand{\StudentName}{Chien-Hsun Chang 張健勳}
\newcommand{\StudentID}{614410073}
\newcommand{\StudentInfo}{\StudentName~(\StudentID)}
\newcommand{\DateDue}{Nov.\ 7, 2025}
\newcommand{\DateHandedIn}{Sep.\ 30, 2025}

\raggedbottom
\renewcommand{\thefootnote}{[\arabic{footnote}]}
\begin{document}

% ---------------- 封面 ----------------
\begin{titlepage}
  \centering
  {\vspace*{2cm}\Huge\bfseries \HomeworkTitle\par}
  \vspace{2cm}
  {\Large \StudentInfo\par}
  \vfill
  \begin{tabular}{@{}ll@{}}
    Date Due: & \DateDue \\
    Date Handed In: & \DateHandedIn \\
  \end{tabular}
  \vfill
\end{titlepage}

% --------------- 內文 -----------------
\section{Technical Description}
本作業使用 Python 實作三種 Spatial Image Enhancement 方法:
(1.1) Power-Law (Gamma) Transformation、
(1.2) Histogram Equalization、
(1.3) Image Sharpening Using the Laplacian Operator。
以下分別介紹三者原理與實作細節,並於(1.4)說明 Implementation。

\subsection{Power-Law (Gamma) Transformation}
Power-Law 變換是一種影像強度映射,其數學表達為
\[
s = c \cdot r^{\gamma},
\]
其中 \(r\) 為輸入像素強度(通常先正規化至 \([0,1]\)),\(s\) 為輸出強度,\(c\) 為縮放常數(常設為 1),\(\gamma\) 為控制參數。其映射特性如下:
\begin{itemize}
  \item \(\gamma < 1\):輸出偏亮,暗部細節被拉伸;
  \item \(\gamma > 1\):輸出偏暗,亮部被壓縮;
  \item \(\gamma = 1\):變換為線性(若 \(c=1\) 則為恆等映射)。
\end{itemize}

Gonzalez 與 Woods 指出,Power-Law 變換常亦稱為 Gamma 變換,用於影像亮度與對比度調整;在實際影像系統中,Gamma 校正(Gamma Correction / Encoding / Decoding)以相同之 Power-Law 形式補償裝置的非線性響應,使輸入與顯示之亮度關係更合理 \cite{GonzalezWoods2018}。

實作步驟範例:
\begin{enumerate}
  \item 正規化:\( r = r_{\text{orig}}/255 \)(8-bit 影像);
  \item 套用:\( s = c \cdot r^{\gamma} \);
  \item 反正規化:\( s_{\text{out}} = \mathrm{clip}(\mathrm{round}(255 \cdot s)) \)。
\end{enumerate}

\subsection{Histogram Equalization}
Histogram Equalization(直方圖均化)透過一個單調遞增的灰階映射,使輸出影像的灰階分佈更趨近均勻,以提升整體對比度,特別對原始影像灰階集中(過暗或過亮)時效果顯著。其核心概念為使用輸入影像的累積分佈函數(CDF, Cumulative Distribution Function)作為新的灰階映射。假設輸入離散灰階為 \(r_k\),對應機率質量函數(PMF) \(p(r_k) = n_k/(MN)\),其中 \(n_k\) 為灰階 \(r_k\) 出現次數,\(M,N\) 為影像尺寸。累積分佈函數為
\[
\mathrm{CDF}(r_k) = \sum_{j=0}^{k} p(r_j).
\]
映射函數(以 8-bit, \(L=256\) 為例)可寫成:
\[
s_k = T(r_k) = \mathrm{round}\left( \frac{\mathrm{CDF}(r_k)-\mathrm{CDF}_{\min}}{MN-\mathrm{CDF}_{\min}} (L-1) \right),
\]
其中 \(\mathrm{CDF}_{\min}\) 為第一個非零 CDF 值,用以避免影像中存在未使用灰階時造成對比壓縮。此即程式 `HistogramEqualizationProcessor.apply_histogram_equalization()` 中的公式:
\begin{enumerate}[label=\alph*)]
  \item 逐像素計算原始 256-bin histogram(以雙層 for-loop 實作,符合「不使用 OpenCV/PIL 現成增強」要求)。
  \item 計算 CDF 與 \(\mathrm{CDF}_{\min}\)。
  \item 依上述公式計算對應新灰階並截斷至 \([0,255]\)。
\end{enumerate}
特點:
\begin{itemize}
  \item 可提升整體動態範圍利用率,增強暗/亮區細節;
  \item 局部對比(local contrast)未必充分提升,特別是影像不同區塊亮度差異極大時;
  \item 易放大雜訊:在原本像素稀疏區域(低頻灰階)會進行拉伸;
  \item 全域方法;若需避免過度增強,可改用 CLAHE(對比限制自適應直方圖均化)但本作業未實作。
\end{itemize}

\subsection{Image Sharpening Using the Laplacian Operator}
Laplacian 為二階微分算子,衡量影像的二階變化,用於偵測快速灰階變化(邊緣)。連續形式:
\[
\nabla^2 f = \frac{\partial^2 f}{\partial x^2} + \frac{\partial^2 f}{\partial y^2}.
\]
離散 2D 中常見 4-鄰域近似:
\[
\nabla^2 f(x,y) \approx f(x+1,y)+f(x-1,y)+f(x,y+1)+f(x,y-1) - 4 f(x,y),
\]
或 8-鄰域加強對角方向:
\[
\nabla^2 f(x,y) \approx \sum_{(i,j)\in \mathcal{N}_8} f(x+i,y+j) - 8 f(x,y).
\]
本作業在 `LaplacianImageSharpener` 中提供兩種 kernel:
\begin{itemize}
  \item 8-connected: \(\begin{bmatrix}-1 & -1 & -1\\ -1 & 8 & -1\\ -1 & -1 & -1\end{bmatrix}\)
  \item 4-connected: \(\begin{bmatrix}0 & -1 & 0\\ -1 & 4 & -1\\ 0 & -1 & 0\end{bmatrix}\)
\end{itemize}
為強調邊緣,採用「原圖加回(unsharp-like)」的策略:
\[
g(x,y) = f(x,y) + \nabla^2 f(x,y),
\]
對應程式 `apply_sharpening_filter()` 中 `enhanced_pixel_value = image + laplacian_filtered_result`,再截斷至 \([0,255]\)。此作法與傳統 unsharp masking(先平滑再加細節)不同,但同樣會增加邊緣對比。注意:
\begin{itemize}
  \item 高頻雜訊同時被放大;
  \item 若需控制強度,可引入比例係數 \(\alpha\): \(g = f + \alpha \nabla^2 f\);
  \item 目前未做負值偏移(如常見的 \(f - c\nabla^2 f\) 形式),因 kernel 中中心為正值。
\end{itemize}

\subsection{Implementation}
\subsubsection{Platform and Packages}
使用 Python(本機假設 3.10+)與基礎科學繪圖套件:`numpy` 進行陣列處理、`Pillow` 進行影像讀寫(僅載入/儲存,不用其增強功能)、`matplotlib` 產生影像與直方圖視覺化。資料驗證結構(若後續擴充)使用 `pydantic`,執行檔打包採 `pyinstaller`(提供 `build_exe.bat` 與 `.spec`)。
依 `requirements.txt`:numpy / pillow / matplotlib / pydantic / pytest / pyinstaller。
\subsubsection{Project Structure}
核心目錄(節錄):
\begin{itemize}
  \item `main.py`:進入點,批次對四張測試影像執行三種增強並儲存結果。
  \item `src/enhancement/`:三個演算法實作(power\_law, histogram\_equalization, laplacian)。皆使用雙層迴圈示範基礎原理,避免黑箱。
  \item `src/pipeline/processing_pipeline.py`:封裝計算、視覺化、儲存流程;`process_single_image()` 形成可重用工作單元。
  \item `src/ui/visualization.py`:產生 2×4 子圖:四張影像 + 四個 histogram,比較增強前後灰階分佈改變。
  \item `src/utils/image_utils.py`:檔案載入、儲存、histogram 統計整合。
  \item `test_image/`:提供四張 8-bit BMP 灰階測試影像(Cameraman, Jetplane, Lake, Peppers)。
  \item `results/`:輸出:`*_gamma.bmp`、`*_hist_eq.bmp`、`*_sharpened.bmp` 及 `*_comparison.png`。
\end{itemize}
\subsubsection{Parameter Settings}
\begin{itemize}
  \item Gamma 變換:預設 \(\gamma = 2.2\)。可調整以測試 \(<1\) 提亮或 \(>1\) 壓暗;本程式集中設定於 `main.py` 的 `gamma_value` 變數。
  \item Histogram Equalization:無額外超參數;採全域均化公式(引入 \(\mathrm{CDF}_{\min}\) 避免空洞灰階造成過度壓縮)。
  \item Laplacian 銳化:預設 8-connected kernel;可改 `'4-connected'`。目前增強強度係數默認為 1(可擴充成可調 \(\alpha\))。
  \item 邊界處理:Laplacian 以「複製邊界像素」(clamped / replicate padding)方式處理索引越界。
\end{itemize}
\subsubsection{Execution Flow}
主要流程(對每張影像):
\begin{enumerate}
  \item 載入灰階影像至 `numpy.float64` 陣列(保留後續運算精度)。
  \item 分別呼叫三個增強函式取得結果:`apply_power_law_transformation`、`apply_histogram_equalization_enhancement`、`apply_laplacian_image_sharpening`。
  \item 視覺化:產生 8 子圖比較並儲存 PNG。
  \item 輸出各增強結果為 BMP(利於無損灰階觀察)。
  \item (可擴充)計算統計指標或客觀品質衡量。
\end{enumerate}
簡化偽碼:
\begin{verbatim}
for img in image_list:
    f = load(img)
    gamma_out = power_law(f, gamma)
    hist_out  = hist_equalize(f)
    lap_out   = laplacian_sharpen(f)
    save(gamma_out, hist_out, lap_out)
    visualize(f, gamma_out, hist_out, lap_out)
\end{verbatim}
\subsubsection{Usage}
以命令列執行:
\begin{verbatim}
python main.py
\end{verbatim}
若需變更 gamma,可直接編輯 `main.py` 中 `gamma_value`。打包(Windows)可使用 `pyinstaller` 與現有 `.spec` 或 `build_exe.bat`。

\section{Experimental Results}
\subsection{Power-Law (Gamma) Transformation Results}
對四張圖(Cameraman, Jetplane, Lake, Peppers)以 \(\gamma=2.2\) 進行處理屬於「壓暗亮部」類型:
\begin{itemize}
  \item 暗部(原直方圖低灰階)變化幅度較小,亮部區域被壓縮,整體對比在高亮區域下降;
  \item 若改用 \(\gamma < 1\) 則會顯著拉伸陰影區,對 Cameraman 的背景細節更有幫助;
  \item 由於原影像多為中低對比,建議實務上測試 \(\gamma=0.5\sim0.9\) 以提升觀察舒適度。
\end{itemize}
\subsection{Histogram Equalization Results}
全域均化後:
\begin{itemize}
  \item Cameraman:臉部與背景陰影細節被拉出,但背景噪點亦被放大;
  \item Jetplane:雲層層次更分明,機翼邊緣更銳利;
  \item Lake:水面反射與山脈層次提升,但天空可能出現輕微條帶(banding);
  \item Peppers:亮色椒面高光壓縮,暗部紋理釋放,整體飽和度主觀感下降(灰階亮度分佈更平均)。
\end{itemize}
\subsubsection{Histograms (before / after)}
視覺化顯示:原始 histogram 可能集中於窄範圍;均化後分佈趨於填滿 0--255,峰值降低、寬度增加。Peppers 具多峰(不同椒段反射),均化後各峰被攤平,提升全域對比但降低某些區域的局部動態層次感。
\subsection{Image Sharpening Results}
Laplacian 銳化後:
\begin{itemize}
  \item 邊緣(高梯度區)亮度提升(或暗邊更暗),視覺上更清晰;
  \item 細微噪聲顆粒亦被強調,特別在 Cameraman 暗背景及 Lake 的水面紋理;
  \item 8-connected kernel 比 4-connected 更強,可能導致 halo 或邊緣過衝(overshoot);
  \item 若觀察到反白邊,可考慮改用 \(g = f - \alpha \nabla^2 f\) 或加入抑制係數。
\end{itemize}
\subsection{Per-Image Observations}
\subsubsection{Cameraman} 中低亮度佔多;Gamma=2.2 使臉部暗沉;Histogram Equalization 拉出臉部紋理與背景雜訊;Laplacian 讓帽邊與肩部更清晰但噪聲增加。
\subsubsection{Jetplane} 原本高亮天空與機體對比適中;Gamma=2.2 壓亮天空使機體對比相對上升;Histogram Equalization 使雲層層次豐富;Laplacian 強化機翼輪廓線。
\subsubsection{Lake} 原圖灰階較平均;Gamma 壓亮區後整體略偏暗;Histogram Equalization 提升山體/水面分界;Laplacian 對樹線與山稜線強化明顯。
\subsubsection{Peppers} 反光與陰影差異大;Gamma=2.2 壓抑高光保留陰影;Histogram Equalization 放大暗部紋理但可能失去柔順漸層;Laplacian 使邊緣輪廓更利、表面顆粒感上升。

\section{Discussions}
\subsection{Comparison of Methods}
三方法分屬不同層面:Power-Law 為強度再映射(非線性曲線),Histogram Equalization 為全域統計再分佈,Laplacian 為空間二階微分濾波(強調高頻)。彼此可串接:例如先 Gamma 校正調整亮度,再均化,最後銳化。順序影響結果(先銳化後均化可能放大噪聲再被重新分佈)。
\subsection{Method Analysis}
\subsubsection{Advantages}
\begin{itemize}
  \item Power-Law:簡單、可連續調整亮暗偏好;適合補償顯示裝置 Gamma。
  \item Histogram Equalization:自動化、不需手動設定門檻;顯著擴展動態範圍使用率。
  \item Laplacian 銳化:計算量低(3×3 kernel),邊緣強化直接。
\end{itemize}
\subsubsection{Disadvantages}
\begin{itemize}
  \item Power-Law:單一參數對不同區域同時作用,無區域自適應能力。
  \item Histogram Equalization:可能過度增強背景噪聲;影像主觀自然度下降。
  \item Laplacian:對噪聲敏感;易產生 overshoot 或 halo。
\end{itemize}
\subsubsection{Limitations}
未實作:CLAHE、雙邊濾波結合銳化、可調銳化強度 \(\alpha\)、噪聲先行抑制(如 Gaussian smoothing)等。Power-Law 目前僅支援灰階 8-bit。
\subsubsection{Noise Amplification}
Histogram Equalization 與 Laplacian 均會放大高頻成分。建議改進:
\begin{itemize}
  \item 先以小 sigma Gaussian 平滑;
  \item Laplacian 使用抑制係數或加權融合:\(g = f + \alpha \nabla^2 f\), \(0<\alpha<1\);
  \item 對均化後影像再行輕度降噪(如 median)以抑制被攤平後的孤立噪點。
\end{itemize}
\subsection{Parameter Sensitivity}
\begin{itemize}
  \item Gamma:對低灰階拉伸靈敏,建議 log 掃描(例如 0.4,0.6,0.8,1.0,1.4,2.2)並以對比度或熵為指標選最優。
  \item Laplacian Kernel:8-connected 會比 4-connected 更強烈;可動態依據影像雜訊估計選擇。
  \item 直方圖均化:無內參,但可評估是否需改用 CLAHE(區塊大小、對比限制 clip limit)。
\end{itemize}
\subsection{Possible Improvements}
\begin{itemize}
  \item 增加 CLAHE 與可調銳化係數介面。
  \item 引入客觀指標(對比度提升率、熵、Tenengrad focus measure 等)。
  \item 管線化與參數化(CLI 參數或 config 檔)。
  \item 增加單元測試:驗證 histogram equalization 映射是否為單調遞增、Laplacian 結果是否總和近零等性質。
  \item 向量化加速:以 `numpy` broadcasting 取代巢狀迴圈(仍可保留教學版本)。
\end{itemize}

\section{References}
\bibliographystyle{IEEEtran}
\bibliography{refs}

\section{Appendix}
\subsection*{A. 主要演算法時間複雜度(理論)}
假設影像尺寸 \(M\times N\):
\begin{itemize}
  \item Power-Law:一次掃描 \(O(MN)\)。
  \item Histogram Equalization:建 histogram + CDF + 映射,共 \(O(MN + L)\),其中 \(L=256\) 固定。
  \item Laplacian:每像素 3×3 kernel \(9\) 次乘加,仍為 \(O(MN)\)。
\end{itemize}
向量化可將 Python 迴圈成本降至 C 實作底層(`numpy`),實際加速倍數視記憶體配置而定。

\subsection*{B. 潛在自動評估指標}
可計算處理前後:全域對比(標準差)、Shannon entropy、邊緣強度(Sobel 能量)、峰值信噪比(若有 ground truth)。

\subsection*{C. 再現性}
全部程式僅依賴 `requirements.txt` 中套件。於 Windows PowerShell:
\begin{verbatim}
pip install -r requirements.txt
python main.py
\end{verbatim}
即可在 `results/` 取得輸出。

\end{document}
