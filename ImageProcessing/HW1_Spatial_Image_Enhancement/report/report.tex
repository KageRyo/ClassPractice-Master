% !TEX program = xelatex
% !TEX encoding = UTF-8
% !BIB program = biber
\documentclass[12pt,a4paper]{article}

% 字體套件
\usepackage{fontspec}
\usepackage[BoldFont]{xeCJK}

% 字體設定
\setCJKmainfont{標楷體}
\setmainfont{Times New Roman}

% 中文自動換行
\XeTeXlinebreaklocale "zh"
\XeTeXlinebreakskip = 0pt plus 1pt

% 封面資訊設定
\newcommand{\HomeworkTitle}{Image Processing Homework \#1: Spatial Image Enhancement}
\newcommand{\StudentName}{Chien-Hsun Chang 張健勳}
\newcommand{\StudentID}{614410073}
\newcommand{\StudentInfo}{\StudentName~(\StudentID)}
\newcommand{\DateDue}{Nov.\ 7, 2025}
\newcommand{\DateHandedIn}{Sep.\ 28, 2025}

\begin{document}

\begin{titlepage} % 封面
	\centering
	{\vspace*{2cm}\Huge\bfseries \HomeworkTitle\par}
	\vspace{2cm}
    {\Large \StudentInfo\par}
	\vfill
	\begin{tabular}{ll}
		Date Due: & {\DateDue} \\
		Date Handed In: & {\DateHandedIn} \\
	\end{tabular}
	\vfill
\end{titlepage}

\section{Technical Description} % 技術說明
  本作業使用 Python 語言實作三種 Spatial Image Enhancement 的影像處理方法,分別為 (1.1)Power-Law (Gamma) Transformation、(1.2)Histogram Equalization 以及 (1.3)Image Sharpening Using the Laplacian Operator。以下將分別介紹這三種方法的原理與實作細節,並說明 (1.4)Implementation。
\subsection{Power-Law (Gamma) Transformation} % 次冪(伽瑪)轉換
  Power-Law Transformation 是一種基於 Power-Law 公式的灰階影像增強方法,其原理為 Gamma Correction,透過調整影像的亮度與對比度來改善視覺效果。其數學表達式為:
\[ s = c \cdot r^{\gamma} \]
  其中,\( s \) 為輸出像素值,\( r \) 為輸入像素值,\( c \) 為常數,而 \( \gamma \) 為調整參數。當 \( \gamma < 1 \) 時,影像會變得較亮;當 \( \gamma > 1 \) 時,影像會變得較暗。此方法適用於調整過曝或欠曝的圖片,在許多系統中(如相機、顯示器等)存在非線性響應(輸入與顯示亮度不是線性關係),gamma 修正就是一種方法,用 \( \gamma \) 來抵消這種非線性,使視覺輸出或顯示更接近線性感知。
\subsection{Histogram Equalization} % 直方圖均化
\subsection{Image Sharpening Using the Laplacian Operator} % 拉普拉斯算子影像銳化
\subsection{Implementation} % 實作細節
\subsubsection{Platform and Packages} % 平台與套件
\subsubsection{Project Structure} % 專案結構
\subsubsection{Parameter Settings} % 參數設定
\subsubsection{Execution Flow} % 執行流程
\subsubsection{Usage} % 使用說明

\section{Experimental Results} % 實驗結果
\subsection{Power-Law (Gamma) Transformation Results}
\subsection{Histogram Equalization Results}
\subsubsection{Histograms (before / after)}
\subsection{Image Sharpening Results}
\subsection{Per-Image Observations}
\subsubsection{Cameraman}
\subsubsection{Jetplane}
\subsubsection{Lake}
\subsubsection{Peppers}

\section{Discussions} % 討論
\subsection{Comparison of Methods} % 方法效果比較
\subsection{Method Analysis} % 方法分析
\subsubsection{Advantages} % 優點
\subsubsection{Disadvantages} % 缺點
\subsubsection{Limitations} % 限制
\subsubsection{Noise Amplification} % 雜訊放大
\subsection{Parameter Sensitivity} % 參數敏感度
\subsection{Possible Improvements} % 改進方向

\section{References and Appendix} % 參考資料與附錄
\subsection{References}
\subsection{Appendix}

\end{document}
