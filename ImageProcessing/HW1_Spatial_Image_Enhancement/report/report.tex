% !TEX program = xelatex
% !TEX encoding = UTF-8
% !BIB program = bibtex
\documentclass[12pt,a4paper]{article}

% --- 版面與中文字 ---
\usepackage{geometry}
\geometry{a4paper,margin=1in}

\usepackage{fontspec}
\usepackage[BoldFont]{xeCJK}
% 統一中英文字體(可按需更換)
\setCJKmainfont{標楷體}
\setmainfont{Times New Roman}

% 中文自動換行+調整 CJK glue,減少不均勻間隙
\XeTeXlinebreaklocale "zh"
\XeTeXlinebreakskip = 0pt plus 1pt
\xeCJKsetup{CJKglue=\hspace{0pt plus 0.08em minus 0.02em}}

% 段落樣式:縮排+微小段前後距,避免空白忽大忽小
\setlength{\parindent}{2em}
\setlength{\parskip}{0.35em}
\linespread{1.2}

% 清單間距統一
\usepackage{enumitem}
\setlist{nosep,leftmargin=2em}

% 數學
\usepackage{amsmath,amssymb}

% --- 參考文獻格式 ---
\usepackage{cite}

% --- 封面資訊 ---
\newcommand{\HomeworkTitle}{Image Processing Homework \#1: Spatial Image Enhancement}
\newcommand{\StudentName}{Chien-Hsun Chang 張健勳}
\newcommand{\StudentID}{614410073}
\newcommand{\StudentInfo}{\StudentName~(\StudentID)}
\newcommand{\DateDue}{Nov.\ 7, 2025}
\newcommand{\DateHandedIn}{Sep.\ 28, 2025}

\raggedbottom
\renewcommand{\thefootnote}{[\arabic{footnote}]}
\begin{document}

% ---------------- 封面 ----------------
\begin{titlepage}
  \centering
  {\vspace*{2cm}\Huge\bfseries \HomeworkTitle\par}
  \vspace{2cm}
  {\Large \StudentInfo\par}
  \vfill
  \begin{tabular}{@{}ll@{}}
    Date Due: & \DateDue \\
    Date Handed In: & \DateHandedIn \\
  \end{tabular}
  \vfill
\end{titlepage}

% --------------- 內文 -----------------
\section{Technical Description}
本作業使用 Python 實作三種 Spatial Image Enhancement 方法:
(1.1) Power-Law (Gamma) Transformation、
(1.2) Histogram Equalization、
(1.3) Image Sharpening Using the Laplacian Operator。
以下分別介紹三者原理與實作細節,並於(1.4)說明 Implementation。

\subsection{Power-Law (Gamma) Transformation}
Power-Law 變換是一種影像強度映射,其數學表達為
\[
s = c \cdot r^{\gamma},
\]
其中 \(r\) 為輸入像素強度(通常先正規化至 \([0,1]\)),\(s\) 為輸出強度,\(c\) 為縮放常數(常設為 1),\(\gamma\) 為控制參數。其映射特性如下:
\begin{itemize}
  \item \(\gamma < 1\):輸出偏亮,暗部細節被拉伸;
  \item \(\gamma > 1\):輸出偏暗,亮部被壓縮;
  \item \(\gamma = 1\):變換為線性(若 \(c=1\) 則為恆等映射)。
\end{itemize}

Gonzalez 與 Woods 指出,Power-Law 變換常亦稱為 Gamma 變換,用於影像亮度與對比度調整;在實際影像系統中,Gamma 校正(Gamma Correction / Encoding / Decoding)以相同之 Power-Law 形式補償裝置的非線性響應,使輸入與顯示之亮度關係更合理 \cite{GonzalezWoods2018}。

實作步驟範例:
\begin{enumerate}
  \item 正規化:\( r = r_{\text{orig}}/255 \)(8-bit 影像);
  \item 套用:\( s = c \cdot r^{\gamma} \);
  \item 反正規化:\( s_{\text{out}} = \mathrm{clip}(\mathrm{round}(255 \cdot s)) \)。
\end{enumerate}

\subsection{Histogram Equalization}
(撰寫直方圖均化之理論與實作)

\subsection{Image Sharpening Using the Laplacian Operator}
(撰寫拉普拉斯算子銳化之理論與實作)

\subsection{Implementation}
\subsubsection{Platform and Packages}
(Python/套件版本等)
\subsubsection{Project Structure}
(專案資料夾與檔案說明)
\subsubsection{Parameter Settings}
(各方法之參數)
\subsubsection{Execution Flow}
(執行流程圖或步驟)
\subsubsection{Usage}
(指令與使用說明)

\section{Experimental Results}
\subsection{Power-Law (Gamma) Transformation Results}
\subsection{Histogram Equalization Results}
\subsubsection{Histograms (before / after)}
\subsection{Image Sharpening Results}
\subsection{Per-Image Observations}
\subsubsection{Cameraman}\subsubsection{Jetplane}\subsubsection{Lake}\subsubsection{Peppers}

\section{Discussions}
\subsection{Comparison of Methods}
\subsection{Method Analysis}
\subsubsection{Advantages}\subsubsection{Disadvantages}\subsubsection{Limitations}\subsubsection{Noise Amplification}
\subsection{Parameter Sensitivity}
\subsection{Possible Improvements}

\section{References}
\bibliographystyle{IEEEtran}
\bibliography{refs}

\section{Appendix}
(如有附錄)

\end{document}
